\documentclass[oneside,a4paper]{amsart}

\usepackage[nopdftoc]{lymber}
\usepackage{enumerate}

\usepackage[T1]{fontenc}
\usepackage{textcomp}
\usepackage{mathpazo}
\usepackage{utopia}

\begin{document}
\titulo{FUVEST 2013 --- Algumas questões}
\subtitulo{Extraído para testar o gerador de provas.}

\thispagestyle{empty}

\vspace{1cm}

\begin{questao}
  Vinte times de futebol disputam a Série A do Campeonato Brasileiro,
  sendo seis deles paulistas.  Cada time joga duas vezes contra cada um
  dos seus adversários. A porcentagem de jogos nos quais os dois
  oponentes são paulistas é
  \begin{enumerate}[\bf a.]
    \item menor que 7\%.
    \item maior que 7\%, mas menor que 10\%. %%% correta
    \item maior que 10\%, mas menor que 13\%.
    \item maior que 13\%, mas menor que 16\%.
    \item maior que 16\%.
  \end{enumerate}
\end{questao}

\begin{questao}
  São dados, no plano cartesiano, o ponto $P$ de coordenadas $(3,6)$ e a
  circunferência $C$ de equação $(x-1)^2+(y-2)^2=1$. Uma reta $t$ passa
  por $P$ e é tangente a $C$ em um ponto $Q$. Então a distância de $P$ a
  $Q$ é

  \begin{enumerate}[\bf a.]
    \item $\sqrt{15}$.
    \item $\sqrt{17}$.
    \item $\sqrt{18}$.
    \item $\sqrt{19}$. %%% correta
    \item $\sqrt{20}$.
  \end{enumerate}
\end{questao}

\begin{questao}
  Os vértices de um tetraedro regular são também vértices de um cubo de
  aresta 2. A área de uma face desse tetraedro é

  \begin{enumerate}[\bf a.]
    \item $2\sqrt{3}$. %%% correta
    \item 4.
    \item $3\sqrt{2}$.
    \item $3\sqrt{3}$.
    \item 6.
  \end{enumerate}
\end{questao}

\begin{questao} 
  Sejam $\alpha$ e $\beta$ números reais com
  $-\frac{\pi}{2}\leq\alpha\leq\frac{\pi}{2}$ e $0\leq\beta\leq\pi$ . Se o
  sistema de equações, dado em notação matricial, \[
  \begin{bmatrix}
    3&6\\
    6&8
  \end{bmatrix}
  \begin{bmatrix}
    \tan\alpha\\\cos\beta
  \end{bmatrix} =
  \begin{bmatrix}
    0\\-2\sqrt{3}
  \end{bmatrix}\] for satisfeito, então $\alpha+\beta$ é igual a

  \begin{enumerate}[\bf a.]
    \item $-\frac{\pi}{3}$.
    \item $-\frac{\pi}{6}$. %%% correta
    \item $0$.
    \item $\frac{\pi}{6}$.
    \item $\frac{\pi}{3}$.
  \end{enumerate}
\end{questao}

\begin{questao}
  As propriedades aritméticas e as relativas à noção de ordem
  desempenham um importante papel no estudo dos números reais. Nesse
  contexto, qual das afirmações abaixo é correta?
  \begin{enumerate}[\bf a.]
    \item Quaisquer que sejam os números reais positivos $a$ e $b$, é
    verdadeiro que $\sqrt{a+b}=\sqrt{a}+\sqrt{b}$.
    \item Quaisquer que sejam os números reais $a$ e $b$ tais que
    $a^2-b^2=0$, é verdadeiro que $a=b$.
    \item Qualquer que seja o número real $a$, é verdadeiro que
    $\sqrt{a^2}=a$.
    \item Quaisquer que sejam os números reais $a$ e $b$ não nulos tais
    que $a<b$, é verdadeiro que $\frac{1}{a}<\frac{1}{b}$.
    \item Qualquer que seja o número real $a$, com $0<a<1$, é verdadeiro
    que $a^2<\sqrt{a}$. %%% correta
  \end{enumerate}
\end{questao}

\begin{questao}
  A extremidade de uma fibra ótica adquire o formato arredondado de uma
  microlente ao ser aquecida por um laser, acima da temperatura de
  fusão. A figura abaixo ilustra o formato da microlente para tempos de
  aquecimento crescentes ($t_1<t_2<t_3$).

  Considere as afirmações:

  \begin{enumerate}[\bf I.]
    \item O raio de curvatura da microlente aumenta com
    tempos crescentes de aquecimento.
    \item A distância focal da microlente diminui com tempos
    crescentes de aquecimento.
    \item Para os tempos de aquecimento apresentados na
    figura, a microlente é convergente.
  \end{enumerate}

  Está correto apenas o que se afirma em

  \begin{enumerate}[\bf a.]
    \item I.
    \item II.
    \item III.
    \item I e III.
    \item II e III. %%% correta
  \end{enumerate}
\end{questao}

\end{document}

%%% Local Variables: 
%%% mode: latex
%%% TeX-master: t
%%% End: 
