\documentclass[oneside,a4paper]{amsart}

\usepackage[nopdftoc]{lymber}
\usepackage{enumerate}

\usepackage[T1]{fontenc}
\usepackage{textcomp}
\usepackage{mathpazo}
\usepackage{utopia}

\begin{document}
\titulo{FUVEST 2013 --- Algumas questões}
\subtitulo{Extraído para testar o gerador de provas.}

\thispagestyle{empty}

\vspace{1cm}

\begin{questao}
  Vinte times de futebol disputam a Série A do Campeonato Brasileiro,
  sendo seis deles paulistas.  Cada time joga duas vezes contra cada um
  dos seus adversários. A porcentagem de jogos nos quais os dois
  oponentes são paulistas é
  \begin{enumerate}[\bf a.]
    \item menor que 7\%.
    \item maior que 7\%, mas menor que 10\%.
    \item maior que 10\%, mas menor que 13\%.
    \item maior que 13\%, mas menor que 16\%.
    \item maior que 16\%.
  \end{enumerate}
\end{questao}

\begin{questao}
  São dados, no plano cartesiano, o ponto $P$ de coordenadas $(3,6)$ e a
  circunferência $C$ de equação $(x-1)^2+(y-2)^2=1$. Uma reta $t$ passa
  por $P$ e é tangente a $C$ em um ponto $Q$. Então a distância de $P$ a
  $Q$ é

  \begin{enumerate}[\bf a.]
    \item $\sqrt{15}$.
    \item $\sqrt{17}$.
    \item $\sqrt{18}$.
    \item $\sqrt{19}$.
    \item $\sqrt{20}$.
  \end{enumerate}
\end{questao}

\begin{questao}
  Os vértices de um tetraedro regular são também vértices de um cubo de
  aresta 2. A área de uma face desse tetraedro é

  \begin{enumerate}[\bf a.]
    \item $2\sqrt{3}$.
    \item 4.
    \item $3\sqrt{2}$.
    \item $3\sqrt{3}$.
    \item 6.
  \end{enumerate}
\end{questao}

\begin{questao} 
  Sejam $\alpha$ e $\beta$ números reais com
  $-\frac{\pi}{2}\leq\alpha\leq\frac{\pi}{2}$ e $0\leq\beta\pi$ . Se o
  sistema de equações, dado em notação matricial, \[
  \begin{bmatrix}
    3&6\\
    6&8
  \end{bmatrix}
  \begin{bmatrix}
    \tan\alpha\\\cos\beta
  \end{bmatrix} =
  \begin{bmatrix}
    0\\-2\sqrt{3}
  \end{bmatrix}\] for satisfeito, então $\alpha+\beta$ é igual a

  \begin{enumerate}[\bf a.]
    \item $-\frac{\pi}{3}$.
    \item $-\frac{\pi}{6}$.
    \item $0$.
    \item $\frac{\pi}{6}$.
    \item $\frac{\pi}{3}$.
  \end{enumerate}
\end{questao}

\end{document}

%%% Local Variables: 
%%% mode: latex
%%% TeX-master: t
%%% End: 
